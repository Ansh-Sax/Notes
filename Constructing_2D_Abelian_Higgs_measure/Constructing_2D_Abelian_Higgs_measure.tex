%%% --------------- DO NOT CHANGE THE FOLLOWING SECTION --------------------
\documentclass[12pt]{article}
\usepackage[margin=1in]{geometry}
\usepackage{amsmath}
\usepackage{amssymb}
\usepackage{amsthm}
\usepackage{graphicx}
\usepackage{hyperref}
\usepackage{tikz}
\usepackage{tikz-3dplot}
\usepackage{pgfplots}
\usetikzlibrary{arrows.meta,calc,3d}
\usepackage{fancyhdr}
\usepackage{multirow, multicol}
\usepackage{enumitem}
\usepackage{tabu}
\usepackage{hyperref}
\usepackage{dsfont}
\usepackage{mathrsfs}
\usepackage{stmaryrd}
\usepackage{mathtools}
\usepackage{cancel}
\usepackage{biblatex}

\pgfplotsset{compat=1.17}
\tdplotsetmaincoords{65}{110}

\newtheorem{defn}{Definition}
\newtheorem{theorem}{Theorem}
\newtheorem{prop}{Proposition}
\newtheorem{lemma}{Lemma}
\newtheorem{corr}{Corollary}
\newtheorem{remark}{Remark}
\newtheorem{example}{Example}
\newtheorem{assmptn}{Assumption}
\theoremstyle{definition}
\newtheorem*{notation}{Notation}
\usepackage{comment}
\newenvironment{solution}{
  \textbf{Solution.}}{}
\renewcommand*{\thefootnote}{\fnsymbol{footnote}\fnsymbol{footnote}}

 \setlength{\headheight}{14.5pt}
 \addtolength{\topmargin}{-2.5pt}


%% ------------------ EXOTIC COMMANDS --------------------------------------

%----------------- Widebar -------------------------------------------------------------------------------------------------------------------------------------------------------------------------------------------------
%See https://tex.stackexchange.com/questions/16337/can-i-get-a-widebar-without-using-the-mathabx-package?noredirect=1&lq=1

\makeatletter
\let\save@mathaccent\mathaccent
\newcommand*\if@single[3]{%
  \setbox0\hbox{${\mathaccent"0362{#1}}^H$}%
  \setbox2\hbox{${\mathaccent"0362{\kern0pt#1}}^H$}%
  \ifdim\ht0=\ht2 #3\else #2\fi
  }
%The bar will be moved to the right by a half of \macc@kerna, which is computed by amsmath:
\newcommand*\rel@kern[1]{\kern#1\dimexpr\macc@kerna}
%If there's a superscript following the bar, then no negative kern may follow the bar;
%an additional {} makes sure that the superscript is high enough in this case:
\newcommand*\widebar[1]{\@ifnextchar^{{\wide@bar{#1}{0}}}{\wide@bar{#1}{1}}}
%Use a separate algorithm for single symbols:
\newcommand*\wide@bar[2]{\if@single{#1}{\wide@bar@{#1}{#2}{1}}{\wide@bar@{#1}{#2}{2}}}
\newcommand*\wide@bar@[3]{%
  \begingroup
  \def\mathaccent##1##2{%
%Enable nesting of accents:
    \let\mathaccent\save@mathaccent
%If there's more than a single symbol, use the first character instead (see below):
    \if#32 \let\macc@nucleus\first@char \fi
%Determine the italic correction:
    \setbox\z@\hbox{$\macc@style{\macc@nucleus}_{}$}%
    \setbox\tw@\hbox{$\macc@style{\macc@nucleus}{}_{}$}%
    \dimen@\wd\tw@
    \advance\dimen@-\wd\z@
%Now \dimen@ is the italic correction of the symbol.
    \divide\dimen@ 3
    \@tempdima\wd\tw@
    \advance\@tempdima-\scriptspace
%Now \@tempdima is the width of the symbol.
    \divide\@tempdima 10
    \advance\dimen@-\@tempdima
%Now \dimen@ = (italic correction / 3) - (Breite / 10)
    \ifdim\dimen@>\z@ \dimen@0pt\fi
%The bar will be shortened in the case \dimen@<0 !
    \rel@kern{0.6}\kern-\dimen@
    \if#31
      \overline{\rel@kern{-0.6}\kern\dimen@\macc@nucleus\rel@kern{0.4}\kern\dimen@}%
      \advance\dimen@0.4\dimexpr\macc@kerna
%Place the combined final kern (-\dimen@) if it is >0 or if a superscript follows:
      \let\final@kern#2%
      \ifdim\dimen@<\z@ \let\final@kern1\fi
      \if\final@kern1 \kern-\dimen@\fi
    \else
      \overline{\rel@kern{-0.6}\kern\dimen@#1}%
    \fi
  }%
  \macc@depth\@ne
  \let\math@bgroup\@empty \let\math@egroup\macc@set@skewchar
  \mathsurround\z@ \frozen@everymath{\mathgroup\macc@group\relax}%
  \macc@set@skewchar\relax
  \let\mathaccentV\macc@nested@a
%The following initialises \macc@kerna and calls \mathaccent:
  \if#31
    \macc@nested@a\relax111{#1}%
  \else
%If the argument consists of more than one symbol, and if the first token is
%a letter, use that letter for the computations:
    \def\gobble@till@marker##1\endmarker{}%
    \futurelet\first@char\gobble@till@marker#1\endmarker
    \ifcat\noexpand\first@char A\else
      \def\first@char{}%
    \fi
    \macc@nested@a\relax111{\first@char}%
  \fi
  \endgroup
}
\makeatother
\newcommand\test[1]{%
$#1{M}$ $#1{A}$ $#1{g}$ $#1{\beta}$ $#1{\mathcal A}^q$
$#1{AB}^\sigma$ $#1{H}^C$ $#1{\sin z}$ $#1{W}_n$}


%------------------------ Laplace Transform --------------------------------
%See https://tex.stackexchange.com/questions/239791/is-this-laplace-transform-symbol-available-in-latex

\newsavebox\foobox
\newlength{\foodim}
\newcommand{\slantbox}[2][0]{\mbox{%
        \sbox{\foobox}{#2}%
        \foodim=#1\wd\foobox
        \hskip \wd\foobox
        \hskip -0.5\foodim
        \pdfsave
        \pdfsetmatrix{1 0 #1 1}%
        \llap{\usebox{\foobox}}%
        \pdfrestore
        \hskip 0.5\foodim
}}
\def\Ell{\slantbox[-.45]{$\mathscr{L}$}}

%% -------------------------------------------------------------------------

%% ------------------------ FORMAT -----------------------------------------

\parindent 0pt
\parskip 10pt

\setcounter{section}{-1}
\pagestyle{fancy}
\hypersetup{
    colorlinks=true,
    linkcolor=blue,
    filecolor=magenta,      
    urlcolor=cyan,
    }
    
% Title
    \title{Integral Calculus over Millenia}
    \date{}
    \author{Ansh S.}
    \addbibresource{bibliography.bib}

% \fancyhead[RO]{MATH-UA 377 Fall 2022}
\fancyhead[RO]{Integral Calculus over Millenia}
\fancyhead[LO]{}
\newenvironment{problems}
{\begin{enumerate}[itemsep=30pt]}
  {\end{enumerate}}
\newenvironment{parts}
{\begin{enumerate}[topsep=10pt,itemsep=20pt,label*=\arabic*.]}
  {\end{enumerate}}


\begin{document}
\maketitle

%% -------------------------------------------------------------------------

%%% -------END OF SECTION THAT MUST BE LEFT UNCHANGED ----------------------


%%% ------------------- SHORTCUTS ------------------------------------------

\newcommand{\R}{\mathbb{R}}
\newcommand{\Z}{\mathbb{Z}}
\newcommand{\Q}{\mathbb{Q}}
\newcommand{\T}{\mathbb{T}}
\newcommand{\Rtilde}{\widetilde{\R}}
\newcommand{\Rvec}{\widehat{\R}}
\newcommand{\id}{\mathds{1}}
\newcommand{\N}{\mathbb{N}}
\newcommand{\C}{\mathbb{C}}
\newcommand{\B}{\mathcal{B}}
\newcommand{\power}{\mathcal{P}}
\newcommand{\M}{\mathcal{M}}
\newcommand{\state}{\mathcal{S}}
\newcommand{\dynkin}{\mathcal{D}}
\newcommand{\A}{\mathcal{A}}
\newcommand{\F}{\mathcal{F}}
\newcommand{\cl}{\mathcal{C}}
\newcommand{\E}{\mathcal{E}}
\newcommand{\g}{\mathfrak{g}}
\newcommand{\no}{\textbf}
\newcommand{\problem}[1]{\textbf{Problem #1}}

\newcommand{\p}{\mathbb{P}}
\newcommand{\e}{\mathbb{E}}
\newcommand{\var}{\operatorname{Var}}
\newcommand{\cov}{\operatorname{Cov}}
\newcommand{\D}{\operatorname{{\bf D}}}
\newcommand{\gauge}{\mathcal{G}}

\newcommand{\proj}{\operatorname{proj}}
\newcommand{\im}{\operatorname{im}}
\newcommand{\tr}{\operatorname{Tr}}
\newcommand{\ric}{\operatorname{Ric}}
\newcommand{\hess}{\operatorname{Hess}}

\let\oldemptyset\emptyset
\let\emptyset\varnothing

%\overset{(d)}{\longrightarrow}

%%% ------------------------------------------------------------------------

\section{Summary}

Bringmann and Cao consider the following Stochastic Abelian-Higgs SPDE:
\begin{align}\label{eq:ym}
    \begin{cases}
        \partial_t  A &= -\D^*_A F_A -B(\D_A\phi,\phi) + \xi\\
        \partial_t\phi &= -\D_A^*\D_A \phi -|\phi|^{q-1}\phi +\zeta
    \end{cases}
\end{align}
where
\begin{align*}
    A&: [0,\infty)\times\T^2\rightarrow \R^2\\
    \phi&: [0,\infty)\times\T^2\rightarrow\C\\
    B&: \C\times\C\rightarrow\R \text{ a certain bi-linear map}\\
    \D_A&: \text{ covariant derivative}\\
    F_A&: \text{ curvature tensor}\\
    \xi,\,\zeta&: \text{ space-time white noise}\\
    q&\geq 3 \text{ an odd integer}.
\end{align*}

Momentarily, so that everything is well defined, consider the following {\it smooth state-space} for $(A,\phi)$ at fixed time:
\begin{align}\label{eq:state}
    \state:= C^\infty(\T^2\rightarrow \R^2)\times C^\infty(\T^2\rightarrow \C).
\end{align}
The {\it group of gauge transformations} is given by
\begin{align}
    \gauge:= \{ g\in C^\infty(\R^2\rightarrow \R) : \nabla g \text{ and } e^{-ig} \text{ are periodic} \},
\end{align}
which is an additive group with addition defined pointwise.

Now $\gauge$ defines a group action on $\state$ defined by
\begin{align}\label{eq:action}
    (A,\phi)\mapsto (A^g,\phi^g):= (A+\nabla g, e^{-ig}\phi).
\end{align}
The physical system under consideration can be in any state $(A,\phi)$. It turns out that all observables of this physical system remain invariant under action of $\gauge$ on the state.
That is, for any state $(A,\phi)$, all elements of the gauge orbit $\{(A^g,\phi^g): g\in\gauge\}$ are physically indistinguishable.
Therefore, the group action given by \eqref{eq:action} is also called the {\it gauge symmetry} of the physical system, and we call any two states in the same gauge orbit {\it gauge equivalent}.

For example, the energy of the system in state $(A,\phi)$ is given by
\begin{align}\label{eq:energy}
    E(A,\phi):= \int_{\T^2} \left( \frac{|F_A|^2}{4} + \frac{|\D_A\phi|^2}{2} + \frac{|\phi|^{q+1}}{q+1} \right).
\end{align}
To understand energy, $\forall (A,\phi)\in\state$, we first define the covariant derivatives $(\D_A^j\phi)_{1\leq j\leq 2}$ and curvature tensor $(F_A^{jk})_{1\leq j,k\leq 2}$ by
\begin{align}
    \D_A^j\phi &:= \partial^j\phi + iA^j\phi,\\
    F_A^{jk} &:= \partial^j A^k - \partial^k A^j.
\end{align}
Now in \eqref{eq:energy}, $|F_A|^2 = F_{A, jk} F_A^{jk}$ and $|\D_A\phi|^2 = \widebar{\D_{A,j}\phi}\D_A^j\phi$ where we follow the Einstein summation convention.
For any $g\in\gauge, (A,\phi)\in\state$, it then follows that
\begin{align*}
    \D_{A^g}^j\phi^g
    &= \partial^j\phi^g + i A^{g,j}\phi^g\\
    &= \partial^j(e^{-ig}\phi) + i(A^j+\partial^j g)e^{-ig}\phi\\
    &= e^{-ig}\partial^j\phi -\cancel{ie^{-ig}\partial^j g  \phi} + \cancel{i e^{-ig}\partial^j g \phi} + ie^{-ig}A^j\phi\\
    &= e^{-ig}[\partial^j\phi + iA^j\phi]\\
    &=e^{-ig}\D_A^j \phi\\
    &= (\D_A^j \phi)^g.
\end{align*}
This gives $\widebar{\D_{A^g,j}\phi^g}=(\widebar{\D_{A,j}\phi})^{-g}$, and hence $|\D_{A^g}\phi^g|^2=|\D_A\phi|^2$. Also,
\begin{align*}
    F^{j,k}_{A^g}
    &= \partial^j A^{g,k} - \partial^k A^{g,j}\\
    &=\partial^j A^k - \partial^k A^j + \partial^j \partial^k g - \partial^k \partial^j g\\
    &= F^{j,k}_A
\end{align*}
which readily implies $|F_{A^g}|^2=|F_A|^2$. Also, trivially, $|\phi^g|^{q-1}=|\phi|^{q-1}$. Thus we find that
\begin{align}
    \forall g\in\gauge, (A,\phi)\in \state,\, E(A^g,\phi^g)=E(A,\phi).
\end{align}

Therefore, it's advantageous to fix a gauge orbit representative in the state space which makes computations the easiest, since the physics doesn't change.
For the purposes of analysing \eqref{eq:ym}, the Coulomb gauge is particularly effective. It sets
\begin{align}\label{eq:coulomb}
    \partial_j A^j = 0
\end{align}
For this choice to be allowed, there must be a unique element in every gauge orbit satisfying \eqref{eq:coulomb}.
For any $A$, we have that $\partial_j A^{g,j} = \partial_j ( A^j + \partial^j g ) = \partial_j A^j + \Delta g$, and thus we reduce to finding $g:\R^2\rightarrow\R$ such that $\nabla g, e^{-ig}$ are periodic and
\begin{align*}
    \Delta g = -\nabla\cdot A.
\end{align*}
This is the Poisson equation. Restricting $g$ to the torus: $g|_{\T^2}$, we see that it will satisfy the above equation iff $\int_{\T^2} -\nabla\cdot A(x)\,dx = 0$, which is always the case by the Divergence theorem.
Next we can extend the solution on the torus $g|_{\T^2}$ periodically to $g:\R^2\rightarrow\R$. Then, $\nabla g$ and $e^{-ig}$ will also be automatically periodic.\\
However, this choice of $g$ is not unique -- $g+h:\R^2\rightarrow\R$ will satisfy the same conditions as long as $\Delta h = 0$ and $\nabla h, e^{-ih}$ are periodic. The following satisfies these properties:
\begin{align}
    \forall n\in\Z^2 \text{ set } h_n(x) = n\cdot x.
\end{align}
In fact, this basically classifies such $h$. To see this, note that $h$ harmonic $\implies \partial^j h$ harmonic, which is also periodic, and therefore by Liouville's theorem $\partial^j h$ must be a constant. Thus $h(x)=a\cdot x+b$ where $a\in\R^2, b\in\R$.
Next we need $e^{ih(x)}=e^{i(a\cdot x+b)}$ to be periodic, which forces $a\in\Z^2$. Thus any such $h:\R^2\rightarrow\R$ will be of the form $n\cdot x+b;\, n\in\Z^2, b\in\R$.\\
Summarily, each gauge orbit has infinitely many states satisfying the Coulomb gauge condition. For such a state $(A,\phi)$, these states are precisely given by $(A^h, \phi^h)$ where $h=n\cdot x+b$.
Since the constant term in $h$ only affects the state by multiplying $\phi$ by a constant, we ignore it. The remaining $h$ then form a subgroup which can then be identified with $\Z^2$:
\begin{align}
    (A,\phi)\mapsto (A^h,\phi^h)=(A+n, e^{-in\cdot x}\phi).
\end{align}
The freedom left in choosing representatives of gauge orbits after imposing a condition is called {\it gauge freedom}. If the condition is stringent enough to identify a unique representative (i.e., removing any gauge freedom), we call the process {\it gauge fixing}.
In the above context, clearly the gauge freedom is discrete and finite dimensional. The former makes it hard to remove it by imposing a further condition, and the latter means it won't be that hard to control the norms we will need for showing well-posedness of \eqref{eq:ym}
in each dimension of the gauge freedom. This is accomplished by considering {\it gauge invariant norms} defined below.

Under the Coulomb gauge condition, the non-linearity in \eqref{eq:ym} exhibits a {\it null structure} which makes the analysis easier.
We modify our state space to always satisfy the Coulomb gauge condition:
\begin{align}\label{eq:statec}
    \state_C:= \{ (A,\phi)\in\state: \partial_j A^j = 0\}.
\end{align}
The equation now reads:
\begin{align}\label{eq:sah}
    \begin{cases}
        \partial_t A &= \Delta A - P_\perp\Im(\widebar{\phi}\D_A\phi) + P_\perp\xi\\
        \partial_t\phi&= \D_A^j\D_{A,j} \phi -|\phi|^{q-1}q +\zeta
    \end{cases}.
\end{align}

As stated above, \eqref{eq:sah} is hard to make sense of due to the low spatial regularity white noise terms. Therefore, we consider a smoothened version of the equation as follows:
\begin{align}\label{eq:sahn}
    \begin{cases}
        \partial_t A_{\leq N} &= \Delta A_{\leq N} -P_\perp\Im(\widebar{\phi_{\leq N}}\D_{A_{\leq N}}\phi_{\leq N}) + C_g A_{\leq N} +P_\perp\xi_{\leq N}\\
        \partial_t\phi_{\leq N} &= \Big(\D_{A_{\leq N}}^j \D_{A_{\leq N}, j} +2\sigma_{\leq N}^2\Big)\phi_{\leq N} - :|\phi_{\leq N}|^{q-1}\phi: +\zeta_{\leq N}
    \end{cases},
\end{align}
where $A_{\leq N}$ and $\phi_{\leq N}$ will be solutions corresponding to the smoothened out noise.
The renormalization term $c_g A_{\leq N}$ (with $C_g=\frac{1}{8\pi}$) is added to make the limiting solution under $N$ {\it gauge covariant} under the discrete gauge transformation \eqref{eq:statec}; it cancels a resonance in the derivative non-linearity.
The other renormalization therm $2\sigma_{\leq N}^2\phi_{\leq N}$ has been added to make the solution finite.
The wick-ordered non-linearity $:|\phi_{\leq N}|^{q-1}\phi_{\leq N}:$ is considered so that the products make sense even at low regularities.

\newpage

The covariant derivatives, unlike the regular derivatives, are non-commutative. From the above definitions, this can be seen as follows:
\begin{align*}
    (\D_A^j\D_A^k - \D_A^k\D_A^j)\phi
    &= \D_A^j (\partial^k\phi + iA^k\phi) - \D_A^k(\partial^j\phi + iA^j\phi)\\
    &= \cancel{\partial^j\partial^k \phi} +iA^j\partial^k\phi + i\partial^j (A^k\phi) - \cancel{A^jA^k\phi}\\
    &\,-\cancel{\partial^k\partial^j\phi} -i A^k\partial^j\phi -i\partial^k (A^j\phi) + \cancel{A^kA^j\phi}
    \intertext{where the terms $\partial^j\partial^k\phi - \partial^k\partial^j \phi$ cancel off by interchanging order of derivatives, and the terms $-A^jA^k\phi + A^kA^j\phi$ cancel off since these are scalar functions that commute.}
    &= i\partial^j(A^k\phi) - iA^k\partial^j\phi\\
    &\, -[i\partial^k(A^j\phi) - iA^j\partial^k\phi]
    \intertext{which simplifies using chain rule:}
    &= i[\partial^j A^k-\partial^kA^j]\phi\\
    &= iF_A^{jk}\phi.
\end{align*}
From this calculation it follows that the curvature tensor captures the non-commutativity of the covariant derivatives.

\end{document}